\documentclass{article}
\usepackage{CJKutf8}
\usepackage{setspace}
\usepackage[fleqn]{amsmath}
\usepackage{amssymb}
\usepackage[pdftex]{graphicx}
\begin{CJK}{UTF8}{gkai}
\title{\textbf{CSAPP Y86 PJ Report}}
\author{Qingfu Wan
\and
Department of Computer Science,Fudan University
}
\begin{document}
\maketitle
\begin{spacing}{1.5}
\section{Introduction}
\noindent
\hangafter=0
\setlength{\hangindent}{2.5em}\\
\begin{Huge}
\textbf{建议用250\%模式查看}\\\\
\end{Huge}
主界面:两个按钮。\\
单击“编辑器”按钮或选择菜单栏“编辑器”进入文本编辑器,单击“处理器”按钮或选择菜单栏“处理器”进入处理器,单击关于进入版本信息界面。
\begin{figure}[htbp]
\centering
\includegraphics[width=1.3\textwidth]{csapppj/pjall.png}
\caption{Overall Form}
\end{figure}
\begin{figure}[htbp]
\centering
\includegraphics[width=0.6\textwidth]{csapppj/pjaboutall.png}
\caption{About All}
\end{figure}
\\
\\
本程序主要实现了以下几个功能:\\
(1)将Y86汇编代码处理成指令编码的形式,即用十六进制指令表示。\\
(2)模拟Y86指令在CPU中的执行,实时显示如书上图4.52的PIPE Hardware Structure,实时显示每条指令流经FDEMW的二维表格(行为指令,列为周期),实时显示历史周期中FDEMW各状态变量的值,实时显示内存数据及栈状态。\\
(3)支持修改CPU频率,控制指令执行速度。\\
(4)支持运行到某一特定周期,支持多断点(指令执行到断电集合中任意一个程序暂停),支持自动运行,支持前进一定指令和后退一定指令。\\
(5)显示代码的编辑器关键词高亮,背景色高亮,在行头显示该指令所处的FDEMW阶段,随着程序执行高亮代码不断变化。\\
\section{About The Program}
关于整个软件\\\\\\\\\\\\\\\\\\
\begin{figure}[htbp]
\centering
\includegraphics[width=1.2\textwidth]{csapppj/about.png}
\caption{Introduction To This Program}
\end{figure}

\section{Form1---Editor}
\subsection{使用说明}
\noindent
\hangafter=0
\setlength{\hangindent}{3.0em}
\begin{normalsize}
\\
单击“文件”菜单,选择“新建”或“打开”,如果当前编辑的代码没有保存,标题栏会显示“无标题- * ”,且会弹出窗口提示未保存。
\begin{figure}[htbp]
\centering
\includegraphics[width=0.2\textwidth]{csapppj/pjeditornew.png}
\caption{Save Before Open Or New File}
\end{figure}
\\
输入一段Y86代码,注意仅在指令如rrmovl和寄存器(立即数)之间有空格,irmovl,rmmovl,mrmovl的立即数用诸如\$0x5F或\$1234表示,call和jXX后面直接跟地址如0xAB0或2345这样的数字表示。\\\\
单击汇编,左图为输入的Y86代码,右边为转换后的指令码,可单击“菜单”--------“保存”按钮保存。
\begin{figure}[htbp]
\centering
\includegraphics[width=0.42\textwidth]{csapppj/pjeditorys.png}
\ \ \ \ \ \ \ \ \ \ \ \ 
\includegraphics[width=0.42\textwidth]{csapppj/pjeditoryo.png}
\caption{Before \& After Assembling---------Convert To Instruction Code}
\end{figure}

\\

\end{normalsize}

\subsection{实现细节}
\noindent
\hangafter=0
\setlength{\hangindent}{8.0em}
\begin{normalsize}
\begin{itemize}

\item
\ \
编辑器用的RichTextBox,由于RichTextBox的高亮只对一行的选中文\\
\indent \ \ 本有效,要实现整行高亮比较困难,于是设计了一个填充颜色为黄色\\
\indent \ \ 的Label控件悬浮在RichTextBox上方,根据光标所在位置移动Label。\\
\indent \ \ 由于整个窗口可能不能显示全部代码,RichTextBox滚动条滑动VScroll\\
\indent \ \ 事件触发行号发生变化,在ListBox中只显示当前窗口内所有行的行号。
\item
\ \
将Y86处理成指令码的过程:\\
\indent \ \ 
把指令符号如rrmovl,立即数或寄存器,逗号,立即数或寄存器用字符\\
\indent \ \ 串处理分割开,然后编码,没什么好说的。
\end{itemize}
\end{normalsize}
\section{Form2---CPU Processor}
\subsection{流水线图表}
\subsubsection{使用说明}

\begin{normalsize}
\begin{itemize}
\begin{figure}[htbp]
\centering
\includegraphics[width=0.5\textwidth]{csapppj/pjdiagram.png}
\caption{Overall Pipeline Diagram}
\end{figure}
\item
单击“全部运行”按钮,程序自动执行,可以看到各个阶段各个状态变量的值自动变化,绿色框中是变量的值,红色框中是当前某个状态寄存器中的指令。
运行完毕后会弹出窗口显示运行完毕时的状态$\left( SHLT,SADR \right) $,三个Go按钮、暂停、继续、停止按钮失效。
\begin{figure}[htbp]
\centering
\includegraphics[width=0.2\textwidth]{csapppj/pjfinish.png}
\caption{Run All}
\end{figure}
\item
单击“重置”按钮后,按钮恢复有效,回到最开始的周期0,按全部运行可以从头开始运行指令。
\item
单击左侧行号可以设置断点,支持多选设置多个断点,再次单击可以取消设置断点。当断点集合中任一指令进入Fetch阶段,程序自动暂停,如下图。断点会随滚动条滚动而自动滚动。左侧FDEMW显示每条指令进行到哪一阶段。
\begin{figure}[htbp]
\centering
\includegraphics[width=0.6\textwidth]{csapppj/pjbreakpoint.png}
\caption{Breakpoint Diagram}
\end{figure}
\item
单击左侧选项卡“设置”可以设置CPU频率,1Hz大概是1秒一条指令。单击“暂停“按钮暂停指令执行,单击“继续”按钮继续执行,单击“停止”按钮停止当前整个Y86指令的执行,如下图。比如按“全部运行”程序停在断点处,再按“继续”或“全部运行”程序继续运行直到下一个断点或程序终止处。
\begin{figure}[htbp]
\centering
\includegraphics[width=0.25\textwidth]{csapppj/pjtimer.png}
\caption{Timer Setting}
\end{figure}
\item
在Address中输入想要读/写的内存地址,单击“读内存”在Value中显示内存在这一地址的值,单击“写内存”将Value赋值给内存Address地址,或者按照文本框中的格式:地址 值,然后单击“设置”按钮将文本框中的信息写进内存。
\begin{figure}[htbp]
\centering
\includegraphics[width=0.4\textwidth]{csapppj/pjmemory.png}
\caption{Memory Read And Write}
\end{figure}
\end{itemize}
\end{normalsize}
\subsubsection{实现细节}
\noindent
\hangafter=0
\setlength{\hangindent}{3.5em}
\\
当内存地址$\leq$20000时,直接存在hash表中,否则存在一个二维数\\
组mapmem中,第一维是地址,第二维是值。本来是想存在类似C++的\\
STL map里面的,考虑到只作演示用,没有太多考虑效率的问题。具体流水线各
寄存器、状态变量值的更新见下文。有了单步执行的代码,运行到某一cycle,前进/后退若干cycle,断点暂停等都是小事。具体使用了RichTextBox控件和左侧两个ListBox控件。
\subsection{简易图}
\noindent
\hangafter=0
\setlength{\hangindent}{2.8em}
简易图里在文本框中显示各状态变量各寄存器的值,文本框只读。
\begin{figure}[htbp]
\centering
\includegraphics[width=0.9\textwidth]{csapppj/pjsimple.png}
\caption{Simple Diagram}
\end{figure}

\subsection{FDEMW}
\subsubsection{使用说明}
\noindent
\hangafter=0
\setlength{\hangindent}{3.5em}
左侧为指令,右侧为指令所经过的阶段,有可能不是全部的FDEMW,上方数字表示对应周期,可以看到每条指令在什么周期进入到了什么阶段。暂停$\left( stall \right)$的F或者D用另一种颜色填充。\\\\
\begin{figure}[htbp]
\centering
\includegraphics[width=1.0\textwidth]{csapppj/FDEMW.png}
\caption{FDEMW}
\end{figure}\\
单击“显示全部”按钮可以看到每条指令执行FDEMW阶段所对应的周期。
\begin{figure}[htbp]
\centering
\includegraphics[width=1.0\textwidth]{csapppj/showall.png}
\caption{showall}
\end{figure}
\subsubsection{实现细节}
\noindent
\hangafter=0
\setlength{\hangindent}{3.5em}\\
为了方便展示,需要记录每个周期FDEMW五个寄存器中的指令分别是什么,需要记录每条指令是否经过了F、D、E、M、W五个阶段以及分别在什么周期处于这些阶段。值得注意的是当遇到加载/使用异常时,F和D分别为stall,上一条指令FDDEMW,下一条指令FFDEMW,到再下一条指令就恢复正常了,即stall只会在前后两条指令(一条load一条use)执行时发生。具体使用了DataGridView控件。
\subsection{Show On ListBox}
\subsubsection{使用说明}
\noindent
\hangafter=0
\setlength{\hangindent}{3.5em}\\
在文本框中输入想要查看的周期数,按回车即可看到输入周期的Fetch、\\
Decode、Execute、Memory、Write Back阶段所有状态变量的值(包括异常状态,所在执行的指令具体是哪条以及一些不太重要的中间变量)、寄存器的值等等。按“$\langle $”回退,“$\rangle $”前进,“$\langle \langle$”设置第一周期,“$\rangle \rangle$”设置最后一个周期,“保存到文件”将每个周期各个变量的值保存到文件,就像pj的ppt要求上写的那样。
\begin{figure}[htbp]
\centering
\includegraphics[width=0.3\textwidth]{csapppj/pjlistbox.png}
\ \ \
\includegraphics[width=0.4\textwidth]{csapppj/pjlistbox2.png}
\caption{ListBox}
\end{figure}
\subsubsection{实现细节}
\noindent
\hangafter=0
\setlength{\hangindent}{2.8em}
只需要用数组保存每个周期各个状态变量的值即可。
\subsection{Memory}
\subsubsection{使用说明}
\noindent
\hangafter=0
\setlength{\hangindent}{2.8em}
在文本框中输入想要显示的内存上界和内存下界,单击“显示”按钮,在列表框中呈现地址和地址对应的值。
\begin{figure}[htbp]
\centering
\includegraphics[width=0.5\textwidth]{csapppj/pjshowmem.png}
\caption{Show Memory}
\end{figure}\\
单击“显示栈”按钮显示从栈底esp往上若干地址元素的值,一个绿色的箭头指向栈底。\\\\
\begin{figure}[htbp]
\centering
\includegraphics[width=0.6\textwidth]{csapppj/pjshowstack.png}
\caption{Show Stack}
\end{figure}
\subsubsection{实现细节}
\noindent
\hangafter=0
\setlength{\hangindent}{3.5em}
简单的ListBox操作。
\subsection{Setting}
\subsubsection{使用说明}
\noindent
\hangafter=0
\setlength{\hangindent}{3.5em}\\
选择“不使用计时器”时,程序自动执行,当选择“使用计时器”时,可设置CPU每秒执行的指令数,即Hz,范围是1-1000Hz,滑动滑动条时,文本框中的频率自动变化,当然也可以手动输入设置频率。
\subsubsection{实现细节}
\noindent
\hangafter=0
\setlength{\hangindent}{3.5em}\\
使用一个timer,不使用计时器时,将其Enabled属性设置为false,否则设置为true,每次触发计时器Tick事件,都判断当前是否在运行程序(只对“全部运行”按钮有效),当单击过“暂停”后,每次计时器计时,Tick事件中什么也不执行,当再单击“继续”后,下次计时器计时触发Tick事件就又会正常执行程序。
\section{Y86 Implementation}
\noindent
\hangafter=0
\setlength{\hangindent}{2.8em}\\
整个Y86执行部分是按照以下机制进行的:
\begin{itemize}
\item
基本执行单位为周期,对于每个周期按照Write Back,Memory,Execute,De\\
code,Fetch部分计算。
\item
当前周期的W由上一周期的M计算得来,如果上一周期M的状态是SBUB(气泡 未执行)或SHLT(halt)或SADR(访问内存地址错误或当前指令地址错误)则W阶段没有做任何事。
\item
类似的,当前周期的M由上一周期的E计算得来,若E非SBUB,SHLT或SADR
\item
当前周期的E由上一周期的D计算得来,需要注意的是如果D\rule[-2pt]{0.2cm}{0.5pt}stall=1即D阶段需要暂停一周期,则向当前E插入一个气泡。
\item
当前周期的D由上一周期的F计算得来,如果F\rule[-2pt]{0.2cm}{0.5pt}stall=1,则向当前D插入一个气泡。
\item
当前周期的F阶段获取的PC由PC预测值F\rule[-2pt]{0.2cm}{0.5pt}predPC和上一周期M\rule[-2pt]{0.2cm}{0.5pt}valA,\\
W\rule[-2pt]{0.2cm}{0.5pt}valM综合决定。
\item
按照从W到M到E到D到F计算的一大好处是算W时可以直接用M阶段各个状态变量值,因为此时这些状态变量并未更新,还是上一周期时候的值,这样执行保证了有序性。相当于是从F推进到D,从D推进到E,从E推进到M,从M推进到W,从W推进到F,各状态变量是通过流水线一点一点向上传递的,本质是一个迭代过程。
\item
计算F、D、E、M、W五大阶段各变量的代码主要参照书上HCL代码,做了一些小改动,总计约400行。
\item
一些重要的细节:\\
(1)当D阶段暂停时不是什么都不做,还是需要Decode valA和valB,根据条件选择forward哪个值到d\rule[-2pt]{0.2cm}{0.5pt}valA和d\rule[-2pt]{0.2cm}{0.5pt}valB。\\
(2)如果某一阶段(如Decode)在当前周期是Bubble状态,即未执行,Dec\\
ode部分的所有相关变量都要强行设为Null(我用的是INTMIN=-2147483648),如果这个错的话可能会引起其它地方出现异常。\\
(3)如何决定程序如何终止?如果当W寄存器状态为SHLT(程序本身自带的halt)或者为SADR(两种情况,一种是访问内存Memory阶段产生的内存地址超过范围出错,一种是Fetch阶段Fetch到一个wrong的PC,即PC地址超过范围),则程序应当终止。\\
(4)有些变量在某些周期是没有用的,比如有些指令 JMP不用ALU,要设为空。
\item
伪代码:
\begin{figure}[htbp]
\centering
\includegraphics[width=1.0\textwidth]{csapppj/pjt1.JPG}
\caption{1}
\end{figure}
\begin{figure}[htbp]
\centering
\includegraphics[width=1\textwidth]{csapppj/pjt2.JPG}
\caption{2}
\end{figure}
\begin{figure}[htbp]
\centering
\includegraphics[width=1\textwidth]{csapppj/pjt3.JPG}
\caption{3}
\end{figure}
\end{itemize}
\\\\\\\\
\section{Test}
\subsection{Test 1:\ \ \ \ Fibonacci(n=7)}
\noindent
\hangafter=0
\setlength{\hangindent}{3.5em}
计算Fibonacci第7项,执行情况如下(结果保存在eax中,eax=13)
\begin{figure}[htbp]
\centering
\includegraphics[width=1.3\textwidth]{csapppj/fibo1.png}
\caption{Fibonacci N=7}
\end{figure}
\\\\\\\\\\\\\\
\subsection{Test 2:\ \ \ \ Fibonacci(n=10)}
\noindent
\hangafter=0
\setlength{\hangindent}{3.3em}\\
计算Fibonacci第10项,执行情况如下(结果保存在eax中,eax=55)
\begin{figure}[htbp]
\centering
\includegraphics[width=1.3\textwidth]{csapppj/fibo2.png}
\caption{Fibonacci N=10}
\end{figure}
\\\\\\\\\\\\\\
\subsection{Test 3:\ \ \ \ asum}
书上4.1.2小节的例子
\noindent
\hangafter=0
\setlength{\hangindent}{3.3em}\\
\begin{figure}[htbp]
\centering
\includegraphics[width=1.3\textwidth]{csapppj/asum.png}
\caption{asum}
\end{figure}
\section{Key Point}
\noindent
\hangafter=0
\setlength{\hangindent}{2.8em}\\
本程序有以下几个亮点:\\
1.自动将汇编语句转换为指令码。\\
2.\textbf{画出了书上完整的PIPE图,能动态显示指令语句流经流水线各阶段的过程,动态显示各变量向上传递的过程。}\\
3.\textbf{支持设置多个断点,精确控制指令执行速度,随时暂停继续终止,前进或后退任意周期,在程序执行过程中,一行的最前面显示这条指令处于F、D、E、M、W中的哪个阶段,调试功能强大。}\\
4.代码编辑器背景高亮,关键字高亮,显示行号。\\
5.以多种方式展现寄存器、状态变量、内存和栈中的值,可以方便查看历史的任意时刻的这些值。\\
6.\textbf{画出了类似书上的梯形图,可以方便的查看当前周期有哪些语句处在什么状态,有哪些状态并没有任何语句在执行,可以完整地画出这样的梯形图。}\\
7.\textbf{在所有程序段中都加入了try...catch(System.Exception err)这样的捕捉异常语句,保证程序在执行过程中若出现运行时错误会自动跳出所在程序块,而不会退出整个程序。}\\
8.布局还可以,虽然没有动画,但是表现形式多样,生动形象,便于理解整个Pipeline的流程。
\section{Fundamentation}
\noindent
\hangafter=0
\setlength{\hangindent}{2.8em}\\
本人有一定的UI和c\#基础,上学期的数据结构pj用c\#调用c++程序实现了一个轨迹分析系统。
\begin{figure}[htbp]
\centering
\includegraphics[width=1\textwidth]{csapppj/pjds1.png}
\caption{ds1}
\end{figure}
\begin{figure}[htbp]
\centering
\includegraphics[width=0.54\textwidth]{csapppj/pjds2.png}
\caption{ds2}
\end{figure}
\begin{figure}[htbp]
\centering
\includegraphics[width=0.54\textwidth]{csapppj/pjds3.png}
\caption{ds3}
\end{figure}
\begin{figure}[htbp]
\centering
\includegraphics[width=0.54\textwidth]{csapppj/pjds4.png}
\caption{ds4}
\end{figure}
\begin{figure}[htbp]
\centering
\includegraphics[width=0.52\textwidth]{csapppj/pjds5.png}
\caption{ds5}
\end{figure}
\begin{figure}[htbp]
\centering
\includegraphics[width=0.52\textwidth]{csapppj/pjds6.png}
\caption{ds6}
\end{figure}
\\ \\ \\ \\
\section{Shortcomings}
\noindent
\hangafter=0
\setlength{\hangindent}{2.8em}\\
存在的缺点和不足有以下几点:\\
1.没有完整的动画,不能动态展现值传递过程。\\
2.流水线图表中的背景图画的线有些乱,覆盖在其上的Label控件布局不够美观。\\
3.汇编器暂不支持jXX+函数名比如jmp .L2这种形式转换为指令码(觉得没有什么必要),如果要保存常量数组到内存中,只能手动输入进行设置(觉得没有什么必要)。\\
4.没有用STL中的map模拟内存存储,觉得耗时费力。\\
5.查看历史状态值不支持查看历史内存中的值。(这样的话要保存二维数组,空间开销会非常大)\\
6.简易图没有仔细布局。\\
7.最多只支持11111条指令,FDEMW选项卡中那个DataGridView最多之能显示1111行*1111列。\\
8.模拟CPU运行Y86指令速度比较慢。\\
9.没有增加新指令,或者模拟Cache机制?乱序执行?(什么鬼)\\
\section{Preview Of All Files}
\noindent
\hangafter=0
\setlength{\hangindent}{2.8em}\\
\begin{table}[htbp]
\centering
\begin{tabular}{llr}  
\hline
文件名 & 功能 & 行数\\ \hline  
frmaboutall.cs & 主界面的帮助——关于 & 26\\         
frmabouteditor.cs & 编辑器的帮助—— & 26\\      
frmeditor.cs & 编辑器(编译Y86码) & 629\\
frmmain.cs & 主界面 & 49\\
\textbf{frmtaby86.cs} &  \textbf{处理器(核心部分)} & \textbf{2669}\\ 
 & 总计 & 3350 \\ \hline
\end{tabular}
\caption{Preview}
\end{table}\\
\section{Experiment Environment}
需要安装.NET Framework 4.5\\\\\\\\\\\\\\\\\\
\begin{figure}[htbp]
\centering
\includegraphics[width=1.3\textwidth]{csapppj/env1.png}
\caption{Computer}
\end{figure}
\begin{figure}[htbp]
\centering
\includegraphics[width=1.3\textwidth]{csapppj/env2.png}
\caption{Visual Studio}
\end{figure}
\begin{figure}[htbp]
\centering
\includegraphics[width=1.0\textwidth]{csapppj/env3.png}
\caption{Setup.exe}
\end{figure}
\section{Feelings}
\noindent
\hangafter=0
\setlength{\hangindent}{3.3em}\\
感受就是感觉流水线部分一开始看起来挺复杂,后来对着书上HCL代码把每种情况都推导模拟测试了一遍,想通了流水线运作的机制,只要确定好顺序,比如W\ $\langle\  -$M,M\ $\langle\  -$E,E\ $\langle\  -$D,D\ $\langle\  -$F,就可以比较轻松地写出代码了。流水线核心部分400行一共写了一晚上,调试了一下午+一晚上,中间有很多非常细微的bug只能通过不断测试才能找到,一开始我通过了助教给的asum,但是测试我自己在ubuntu32位系统下生成的求fibonacci第n项的Y86码时,eax返回值一直是错的,最后通过单步调试和静态查错找到了漏洞。另外需要说的是即使单步调试找到了错误,最好再对着代码把整个框架在脑子里模拟一遍,模拟每一种情况下程序会给变量赋什么样的值,执行怎样的if语句,这一步是非常有用的,因为很多调试需要很久的错误可以通过仔细静态查错找出来。另外就是UI部分虽然没有什么技术含量,但是耗时非常多,细节繁琐,需要考虑的问题非常多。整个Y86程序流水线部分只写了一天,调试了一天,UI部分写了6天左右。从代码长度上看,Y86只占411行,UI占了大部分。
总的来说,写完以后对流水线的认识谈不上突破天际,也有了很大的提升,比单纯看书本学到了更多实践知识。
\end{spacing}
\end{document}
\end{CJK}{UTF8}{gkai}
